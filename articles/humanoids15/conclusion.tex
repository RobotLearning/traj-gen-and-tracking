\section{CONCLUSION AND FUTURE WORK}\label{conclusion}

In this paper we presented a novel Iterative Learning Control algorithm that learns to track the goal states of movement primitives. Movement primitives have the advantage that they can start from different initial conditions. The movement primitives (DMPs) are generated in the joint space of the robot and enable the robot to execute optimal striking motions. Control inputs are updated after each trial by using a Newton-Raphson based update rule on the deviations from the goal state of the DMPs.  We show two experiments where we evaluate the performance of the approach: putting in golf and striking in robotic table tennis. 

Movement primitives are used in our approach as a way to end up at a fixed hitting goal state within the desired time period. DMPs can also be extended easily to new goal positions. We are currently evaluating new ways to extend our current framework where we can generalize the learning performance of ILC between DMPs that reach different hitting states.

Finally we think that with our method, a stochastic approach to guide exploration is missing, and it is this that restricts the generalization ability of the proposed algorithm. By extending our framework to include the Probabilistic Movement Primitives~\cite{Paraschos13} we hope to increase the generalization ability of ILC and learn to track a whole \emph{distribution} of movement primitives.

%\section*{APPENDIX}
%
%Appendixes should appear before the acknowledgment.
%
%\section*{ACKNOWLEDGMENT}
%
%The preferred spelling of the word ÒacknowledgmentÓ in America is without an ÒeÓ after the ÒgÓ. Avoid the stilted expression, ÒOne of us (R. B. G.) thanks . . .Ó  Instead, try ÒR. B. G. thanksÓ. Put sponsor acknowledgments in the unnumbered footnote on the first page.