\section{APPENDIX}\label{appendix}

Here we briefly show that the solution $\joint(t)$ to the optimal control problem posed in \eqref{costFnc1} is a third order polynomial for each degree of freedom, $i = 1, \ldots, n$. Consider the following optimal control problem
%
\begin{align}
&\min_{\sysInput,T} \int\limits_{0}^{T}\sysInput^{\mathrm{T}}\vec{R}\sysInput(t) \ \mathrm{d}t, \label{costFnc2} \\
\textrm{s.t. \ } &\ddot{\joint}(t) = \sysInput(t), \label{diffEq1} \\
& \kin_p(\joint(T)) = \ballEst(T), \\
& \kin_n(\joint(T)) = \normal_{\mathrm{des}}(T), \\
&\jac(\joint(T))\dot{\joint}(T) = \racketVel_{\mathrm{des}}(T), \\
&\joint(0) = \joint_{0}, \dot{\joint}(0) = \dot{\joint}_{0}, 
\end{align}
%
The differential equation \eqref{diffEq1} can be written in first order form as 
%
\begin{align}
\dot{\tilde{\joint}}(t) = \underbrace{\begin{bmatrix}\vec{0} & \vec{I} \\ \vec{0} & \vec{0} \end{bmatrix}}_{\vec{A}}\tilde{\joint} + \underbrace{\begin{bmatrix}
\vec{0} \\ \vec{I} \end{bmatrix}}_{\vec{B}}\sysInput,
\end{align}
%
\noindent for $\tilde{\joint} = [\joint^{\mathrm{T}}, \dot{\joint}^{\mathrm{T}}]^{\mathrm{T}} \in \mathbb{R}^{2n}$. Using the maximum principle, the Hamiltonian 
%
\begin{align}
\mathcal{H}(\tilde{\sysInput},\tilde{\joint},t) = \sysInput^{\mathrm{T}}\vec{R}\sysInput(t) + \momenta^{\mathrm{T}}(t)\big(\vec{A}\tilde{\joint} + \vec{B}\sysInput), 
\end{align}
%
\noindent for the momenta $\momenta(t) = [\momenta_1^{\mathrm{T}}(t), \momenta_2^{\mathrm{T}}(t)]^{\mathrm{T}}, \ \momenta_1, \momenta_2 \in \mathbb{R}^{n}$ is minimized at 
%
\begin{align}
\sysInput^{*}(t) = -\frac{1}{2} \vec{R}^{-1}\vec{B}^{\mathrm{T}}\momenta^{*}(t).
\label{HamiltonianMaxInput}
\end{align}
%
Costate equation for the momenta gives
%
\begin{align}
\dot{\momenta}^{*}(t) = \vec{A}^{\mathrm{T}}\momenta^{*}(t),
\end{align}
%
\noindent or in other terms, $\dot{\momenta}^{*}_1 = 0, \dot{\momenta}^{*}_2 = \momenta_1^{*}(t)$. Plugging it into \eqref{HamiltonianMaxInput} and using the state equation \eqref{diffEq1} we get
%
\begin{align}
\ddot{\joint}^{*}(t) = -\frac{1}{2}\vec{R}^{-1}\big(\vec{a}_1t + \vec{a}_2),
\end{align}
%
\noindent which shows that the optimal accelerations are linear functions of time. The joint positions $\joint(t)$ are then third order polynomials with $4n$ coefficients to be determined using $\joint_0, \dot{\joint}_0$ and $\normal_{\mathrm{des}}(T), \ballEst(T), \racketVel_{\mathrm{des}}(T)$ at free final time $T$ as another variable. The transversality condition resulting from the joint position, velocity and time boundary constraints can be written as
%
\begin{align}
\begin{bmatrix}
\mathcal{H}(T) \\
-\momenta(T)
\end{bmatrix} &= D\vec{\Psi}^{\mathrm{T}}\vec{\mu} = \begin{bmatrix}
D_{T}\vec{\Psi} & D_{\joint}\vec{\Psi} & D_{\dot{\joint}}\vec{\Psi}
\end{bmatrix}^{\mathrm{T}}\begin{bmatrix}
0 \\ \mu_{1:9}
\end{bmatrix}, \\
\vec{\Psi} &= \begin{bmatrix}
0 \\ \kin_{p}(\joint(T)) - \ballEst(T) \\ \kin_n(\joint(T)) - \normal_{\mathrm{des}}(T) \\ \jac(\joint(T))\dot{\joint}(T) - \racketVel_{\mathrm{des}}(T)
\end{bmatrix},
\label{transversality}
\end{align}
%
\noindent for some $\vec{\mu} \in \mathbb{R}^{10}$. \eqref{transversality} supplies the additional $4n + 1 - 2n - 9 = 2n - 8$ equations to determine all the variables. A nonlinear equation solver, e.g. $\mathtt{fsolve}$ in MATLAB can be used for this purpose. The optimization \eqref{costFnc3} can be seen as a direct way to solve this problem when inequality constraints are additionally imposed. 

\section*{Joint limits}

The joint position, velocity and acceleration limits $\theta_{\mathrm{MIN}}$, $\dot{\theta}_{\mathrm{MAX}}$, $\ddot{\theta}_{\mathrm{MAX}}$ for the Barrett WAM are shown in Table~\ref{joint limits} where the shoulder joints are noted as SFE, SAA, HR, elbow joint as EB and the wrist joints as WR, WFE and WAA. The range of allowed velocities and accelerations is symmetric, i.e. $\dot{q}_i \in [-\dot{\theta}_{\mathrm{MAX}},\dot{\theta}_{\mathrm{MAX}}]$ and $\ddot{q}_i \in [-\ddot{\theta}_{\mathrm{MAX}},\ddot{\theta}_{\mathrm{MAX}}]$, $i = 1, \ldots, 7$.

\begin{table}
\renewcommand{\arraystretch}{1.3}
\caption{Joint Limits}
\label{joint limits}
\centering
\begin{tabular}{c|c|c|c|c}
& \bfseries $\theta_{\mathrm{MAX}}$ & $\theta_{\mathrm{MIN}}$ & \bfseries $\dot{\theta}_{\mathrm{MAX}}$ & \bfseries $\ddot{\theta}_{\mathrm{MAX}}$ \\
\hline
SFE & $2.60$ & $-2.60$ & $200$ & $200$ \\
\hline
SAA & $2.00$ & $-2.00$ & $200$ & $200$ \\
\hline
HR & $2.80$ & $-2.80$ & $200$ & $200$ \\
\hline
EB & $3.10$ & $-0.90$ & $200$ & $200$ \\
\hline
WR & $1.30$ & $-4.80$ & $200$ & $200$ \\
\hline
WFE & $1.60$ & $-1.60$ & $200$ & $200$ \\
\hline
WAA & $2.20$ & $-2.20$ & $200$ & $200$
\end{tabular}
\end{table}
