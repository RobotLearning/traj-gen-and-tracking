\section{ALGORITHM}\label{alg}

In this section we give the details of the nonlinear constrained optimization~\eqref{costFnc1}. We include additionally the joint position, velocity and acceleration constraints $\jointMax^{(j)}, \jointMin^{(j)}$, $j = 0, 1, 2$ in the optimization as inequality constraints. 

As explained in the previous section, we parametrize third order polynomials in joint-space for each degree of freedom of the robot. That is, along with the striking time $T$ as a free parameter, we have a $2n+1$ dimensional problem with nonlinear equality and inequality constraints. Rewriting~\eqref{costFnc1} in terms of these free parameters with $\vec{R}$ as the identity matrix, we get the following optimization problem
%
\begin{align}
\min_{\joint_f,\dot{\joint}_f, T} & \, 3T^3 \vec{a}_3^{\mathrm{T}}\vec{a}_3 + 3T^2 \vec{a}_3^{\mathrm{T}}\vec{a}_2 + T\vec{a}_2^{\mathrm{T}}\vec{a}_2 \label{costFnc2} \\
\textrm{s.t. \ }
& \kin(\joint_f) = [\normal_{\mathrm{des}}(t),\ballEst(t)], \\
&\jac(\joint_f)\dot{\joint}_f = \racketVel_{\mathrm{des}}(t), \\
& \jointMin \leq \joint_{\mathrm{strike}}(\vec{t}_{\mathrm{ext}}^{i}) \leq \jointMax, \, i = 1,2, \\
& \jointMin \leq \joint_{\mathrm{return}}(\vec{t}_{\mathrm{ext}}^{i}) \leq \jointMax, \, i = 3,4,
\end{align}
%
\noindent where the coefficients of the polynomial $\joint_{\mathrm{strike}}(t) = \vec{a}_3 t^3  + \vec{a}_2 t^2 + \dot{\joint}_0 t + \joint_0$ are related to the parameters as 
%
\begin{align}
\vec{a}_3 &= \frac{2}{T^3}(\joint_0 - \joint_f) + \frac{1}{T^2}(\dot{\joint}_0 + \dot{\joint}_f), \\
\vec{a}_2 &= \frac{3}{T^2}(\joint_f - \joint_0) - \frac{1}{T}(\dot{\joint}_f + 2\dot{\joint}_0).
\label{coeffs}
\end{align}
%
For the return polynomial $\joint_{\mathrm{return}}(t) = \vec{\tilde{a}}_3 t^3  + \vec{\tilde{a}}_2 t^2 + \dot{\joint}_0 t + \joint_0$ the coefficients $\vec{\tilde{a}}_3, \vec{\tilde{a}}_2$ are as in \eqref{coeffs} but with $\joint_0, \joint_f$ and $\dot{\joint}_0, \dot{\joint}_f$ reversed. The joint extrema candidates are checked at times
%
\begin{align}
\vec{t}_{\mathrm{ext}}^{1,2}(j) = \frac{-\vec{a}_2(j) \rpm \sqrt{\vec{a}_2^2(j) - 3\vec{a}_3(j)\dot{\joint}_0(j)})}{3\vec{a}_3(j)}, \, j = 1, \ldots, n,\\
\vec{t}_{\mathrm{ext}}^{3,4}(j) = \frac{-\vec{\tilde{a}}_2(j) \rpm \sqrt{\vec{\tilde{a}}_2^2(j) - 3\vec{\tilde{a}}_3(j)\dot{\joint}_f(j)})}{3\vec{\tilde{a}}_3(j)}, \, j = 1, \ldots, n,
\label{jointPosExtrema}
\end{align}
%
\noindent to make sure we satisfy the joint limits both for the strike trajectory, $i = 1,2$ and for the return trajectory, $i = 3,4$. We clamp the values $\vec{t}^{1,2}_{\mathrm{ext}}$ to the interval $[0, \, T]$ and $\vec{t}^{3,4}_{\mathrm{ext}}$ to $[T, \, T + T_{\mathrm{return}}]$ if they are imaginary or outside their corresponding intervals. Joint velocity and acceleration constraints, although not shown in \eqref{costFnc2}, are enforced in a similar way.

Gradients of the cost function~\eqref{costFnc2} can be easily calculated and fed to a constrained nonlinear optimizer, e.g. a sequential quadratic programming (SQP) based solver. We can run this optimizer whenever we have enough ball samples available to estimate the ball coming towards the robot. One simple solution that we like is to stop the estimation procedure whenever the expected time to pass over the table $T_{\mathrm{table}}$ is less than a maximum threshold $T_{\mathrm{max}}$. The full trajectory generation framework is summarized in Algorithm ~\ref{alg1}.

\begin{algorithm}[tb]
   \caption{OPTIMAL TABLE TENNIS ($\alg$)}
   \label{alg1}
\begin{algorithmic}
   \STATE {\bfseries Input:} $T_{\mathrm{max}} > 0, T_{\mathrm{return}}$ 
   \STATE Initialize EKF
   \STATE Train GP models from demonstrations
   \STATE Move to $\joint_0$
   \REPEAT 
	   \STATE Predict $\ballEst(t)$ with EKF till $T_{\mathrm{table}} < T_{\mathrm{max}}$
	   \STATE Predict optimal $\racketVel_{\mathrm{des}}, \normal_{\mathrm{des}}$ using GP
	   \STATE Compute optimal $\joint_f, \dot{\joint}_f, T$ with SQP
	   \STATE Form polynomial $\joint(t), 0 \leq t \leq T$
	   \STATE Check for constraints, if feasible move
	   \STATE Update GP with $\ballLand$	
   \UNTIL no more balls
\end{algorithmic}
\end{algorithm}