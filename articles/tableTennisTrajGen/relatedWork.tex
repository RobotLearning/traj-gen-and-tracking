% this doesn't fit here so well
One of the most popular frameworks in table tennis is the Virtual Hitting Plane (VHP) method, which is based on the virtual hitting point hypothesis~\cite{Ramanantsoa94}. In this approach, the trajectory of the incoming ball is estimated and filtered from a stream of ball position estimates. Usually a physical flight model is used to predict the intersection point of the ball trajectory with an appropriately chosen plane along the vertical axis. This procedure determines the striking time as well as the striking point. The remaining task-space parameters, the desired racket velocity and normal at striking time, are determined by running the physical flight model backwards from a desired ball landing position and velocity and inverting the ball-racket contact model.

For safety reasons, a minimum hitting time $T_{\mathrm{min}}$ is also specified in addition to the coordinates of the hitting plane. The estimation process has to be terminated at least $T_{\mathrm{min}}$ before the expected hitting time, to give the robot enough reaction time. This prevents high accelerations and any risk of damage to the robot. For more details see \cite{Muelling13}, or \cite{Matsushima05} for a more general discussion. A clear limitation of the method is shown in Figure~\ref{mainIdea}. A player fixing the VHP may not generate feasible trajectories for some ball trajectories. By means of trajectory optimization we generate trajectories that are not constrained to a hitting plane.

A related dynamic framework involving moving targets is the ball catching robot of~\cite{Baeuml11} where a computationally demanding optimization problem is solved online. Analyzed in more detail in~\cite{Baeuml10}, it includes also the catching time as another parameter to be optimized. The framework of~\cite{Kim10} considers generating catching movements for more general objects. Another application of optimal control showing the benefits of spatio-temporal optimization is given in~\cite{Nakanishi2016} on a brachiating robot. The computed solutions require much less control commands when compared with traditional optimal control approaches fixing the time interval. 
% more refs especially the polynomial generation framework cited in Billard's paper
%Robustness of a paddling task is analyzed in~\cite{Burridge99}. 

In the remainder of this paper, we describe the trajectory generation framework in detail. First we introduce previous work on table tennis and other relevant trajectory generation frameworks. In Section~\ref{method} we formalize robot trajectory generation as a specific optimal control problem and incorporate physical modeling based on actual data. In Section~\ref{alg} we introduce an optimization approach for optimizing the previously introduced cost functional under additional constraints. In Section~\ref{results} we evaluate the performance of this approach and compare it with previous inverse kinematics (IK) approaches. Detailed verifications in our realistic simulated table tennis platform are given. We discuss how we can extend our simulations to actual experiments in our robotic platform. Finally, in the conclusions we mention several promising extensions in this regard, which might be necessary for successful table tennis performance. %VHP-based

%The determination of a hitting time can be arbitrary, and the algorithms involving a VHP can lead to unnecessarily restrictive strokes. 

% cite Yanlong's paper