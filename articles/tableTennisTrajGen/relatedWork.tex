\section{RELATED WORK}\label{relatedWork}

% this doesn't fit here so well
We will summarize the Virtual Hitting Plane (VHP) framework here only very briefly. In this approach, the trajectory of the incoming ball is estimated and filtered from a stream of ball position estimates. Usually a physical flight model is used to predict the intersection point of the ball trajectory with a comfortably chosen plane along the vertical axis. This procedure determines the striking time as well as the striking point. The remaining task-space parameters, the desired racket velocity and normal at striking time, is determined by running the physical flight model backwards from a desired ball landing position and velocity and inverting the ball-racket contact model.

For safety reasons, a minimum hitting time $T_{\textrm{min}}$ is also specified in addition to the coordinates of the hitting plane. The estimation process has to be terminated at least $T_{\textrm{min}}$ before the expected hitting time, to give the robot enough reaction time. This prevents high accelerations and any risk of damage to the robot. For more details see \cite{Muelling13}, or \cite{Matsushima05} for a more general discussion. 

%The determination of a hitting time can be arbitrary, and the algorithms involving a VHP can lead to unnecessarily restrictive strokes. 

% cite Yanlong's paper