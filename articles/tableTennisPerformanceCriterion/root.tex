% Temporarily as a seperate document
% Will be placed inside the IROS root latex file

\documentclass[10pt,a4paper]{article}
\usepackage[latin1]{inputenc}
\usepackage{amsmath}
\usepackage{amsthm}
\usepackage{amsfonts}
\usepackage{amssymb}
\usepackage{mathtools}
\usepackage{graphicx}
\usepackage{color}
\usepackage{hyperref}

% theorem environment
\newtheorem{prop}{Proposition}[section]
\newtheorem{theorem}{Theorem}[section]
\newtheorem{prop2}{Proposition}
\newtheorem{lem}{Lemma}
\newtheorem{ex}{Example}

% For algorithms
\usepackage{algorithm}
\usepackage{algorithmic}

% custom commands
\newcommand\at[2]{\left.#1\right|_{#2}} % the at differential sign
\newcommand\scalemath[2]{\scalebox{#1}{\mbox{\ensuremath{\displaystyle #2}}}} % scaling matrices

%% custom macros
\newcommand{\todo}{\textcolor{red}{TODO}} % TODO!
\newcommand{\kin}{\mathcal{T}} % used to denote inverse kinematics
\newcommand{\invKin}{\mathcal{T}^{-1}} % used to denote inverse kinematics

\newcommand{\joint}{q} % used to denote robot state in joint space
\newcommand{\state}{\bar{\joint}} % denotes the generalized coordinates - joint space and velocity coordinates
\newcommand{\dmp}{s} % used to denote the dmp trajectory states
\newcommand{\error}{e} % difference between state and reference
\newcommand{\traj}{r} % used to denote the points on the trajectory to be tracked

\newcommand{\dist}{\epsilon} % denotes the disturbances acting on the rigid body dynamics
\newcommand{\linDist}{d} % denotes the disturbances on the LTV model

\newcommand{\sysInput}{u} % used to denote the system inputs
\newcommand{\linInput}{\tilde{u}} % denotes the LTV inputs
\newcommand{\trjInput}{\nu} % denotes the inputs on the trajectory


% % % % DMP terminology % % % %
\newcommand{\fullvec}{\psi} % full vector for state-ref-dmp-goal
\newcommand{\force}{f} % forcing term of the dmps
\newcommand{\phase}{x} % phase of the dmp
\newcommand{\weights}{w} % weights of the dmp
\newcommand{\basis}{\phi} % basis functions of the dmp as a matrix

% % % % ILC terminology % % % %
\newcommand{\qmatrix}{\Gamma} % denotes the filtering qmatrix term of Bristow et al.
\newcommand{\lmatrix}{L} % denotes the learning matrix of Bristow et al.

\newcommand{\observations}{\mathbf{y}} % used for the observed output
\newcommand{\dynamics}{f}
\newcommand{\dynamicsNominal}{f_{\mathrm{nom}}}
\newcommand{\policy}{\mathbf{\pi}}
\newcommand{\ValueFunction}{J}
\newcommand{\episode}{k} % used for episode number

\newcommand{\totalTime}{T} % total time duration 
\newcommand{\numSteps}{N} % total number of time steps
\newcommand{\numepisode}{K} % total number of episodes

\newcommand{\threshold}{\epsilon}
\newcommand{\alg}{\emph{wILC}}
\newcommand{\dataset}{E}

% Set the paths where all figures are taken from:
\graphicspath{{Pictures/}}
\mathtoolsset{showonlyrefs} 
\newcommand{\includesvg}[1]{%
% \executeiffilenewer{#1.svg}{#1.pdf}%
% {inkscape -z -D --file=#1.svg %
% --export-pdf=#1.pdf --export-latex}%
 \input{#1.pdf_tex}%
}

\author{Okan Ko\c c, Guilherme Maeda, Gerhard Neumann and Jan Peters}
\title{Optimal Striking Movement Representation \& Execution}
\begin{document}
\maketitle


\section{Robust Control Applications to Table Tennis}

\subsection{Game Theoretic Analysis}

What actions are available to players? Position and body orientation with racket position and body orientation
(initial gesture) and hitting motions (trajectories) as dynamic movement primitives.

Timing of the interactions - sequential model. But not always at the same time

What information do different players have when they take actions? 
They see the opponent's posture always and also their hitting motion. They will also see the incoming ball trajectory - but maybe not everything if it's coming too fast. They may or may not know the spin of the ball.
Sometimes you have to choose, or take action, before seeing the opponent hit the ball.

What are the payoffs to the various players as a result of the interaction?
For the players there are no payoffs unless the ball does not touch the table on the other side. In that case, one gets 1 and the other gets 0.

Sensitivity analysis - am I acting strategically enough?

Pure strategies over actions vs. mixed strategies

Dominant strategy - an individual does not have to make any predictions about what other players
might do, and still has a well-defined best strategy. This adds robustness.

A pure strategy Nash equilibrium only requires that the
action taken by each agent be best against the actual equilibrium actions taken by the other
players, and not necessarily against all possible actions of the other players.
"we have to look for a pair of actions such that neither firm wants
to change its action given what the other firm has chosen"

A profile of dominant strategies is a Nash equilibrium but not vice versa.

randomized strategies, or what are called mixed strategies, when no pair of pure strategies forms an equilibrium.
maximize entropy - do not reveal info to the opponent, he shouldn't be able to predict your future behaviour given past samples.

is table tennis an extensive form game? Game tree? backwards induction 

Spring-like dynamics of the game. You throw hoping that it will never come back (zero disturbance from opponent) but if there is enough disturbance then it will come back. 

The table in table tennis defines the stability region of the space. Precise terminology of the stability region?

\bibliographystyle{plain}
%\bibliographystyle{./IEEEtran}
%\bibliography{./IEEEabrv,./iros2015Ref}
\bibliography{./tempRef}

\end{document}
